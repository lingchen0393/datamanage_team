\documentclass[a4paper,11pt]{article}
\usepackage{color,xcolor,ucs}
\usepackage[top=1.2in, bottom=1.2in, left = 1in, right = 1in]{geometry}
\usepackage[linkcolor=black,colorlinks=true,urlcolor=blue]{hyperref}
\usepackage{indentfirst}
\setlength{\parindent}{2em}
\usepackage[section]{placeins}
\usepackage{float}

\usepackage{amsmath}
\title{Initial Report}
\author{datamanage\_team}

\begin{document}
\maketitle
\begin{center}
\rule{\textwidth}{1pt}
\end{center}

\noindent 
\section{Project Aims}
Recently there have been many excellent file synchronization tools(e.g. Dropbox, Unison) that make our work easier and more harmonized. Our goal in this class is to develop a similar multi-host file synchronizer which could help make it possible to edit the same files across multiple computers. Our project will adopt “hub and spoke” model that allow multiple users to sync files using web pages and Android apps. According to our objectives, the main functions of our software can be divided into six parts:
\begin{enumerate}
\item User Management: including login/logout, registration, information modification
\item Private file synchronization: users can upload, download, move, rename, delete files in their own private folders.
\item  Public file synchronization: public folders allow multiple users to sync files.
\item Conflict resolution: resolve renaming conflicts and conflicts caused by failure to synchronize in time
\item Security issue: using MD5 Message-Digest Algorithm to encrypt user’s password and storing in the database; using the RSA algorithm to encrypt the transmitted data.
\item Compressed transmission: using RSync algorithm to make transmissions more efficient and save bandwidth.
\end{enumerate}
\section{Strategy}
\subsection{Agile Software Development}
There are various sorts of development method in software development nowadays. As a rookie team which was just set up, we decide to adopt Agile as our development method, for it is flexible and elastic. The most significant point is that Agile allows the team to develop iteratively, instead of confirming the whole process involved design and implementation at the first beginning without any opportunity to update.
\subsection{Scrum}
Among different agile frameworks, we introduce the method called "Scrum", which is designed for teams of three to nine members just like our team. What a scrum team to do is to break whole development into timeboxed iterations, called "sprints", normally between one month to two weeks, then track progress and re-plan in everyday meetings, also called daily scrums. 
We summarized all the product requirements in product backlog, sorted with priority. In each sprint, we choose one or several requirements to achieve(number of selected requirements is based on difficulty). The daily meeting is achieved by face-to-face meeting or applications including Paper by Dropbox and Wechat.
The idea is developing in an incremental way, updating the product once and once again and eventually complete in a highly-integrated way.
\section{Backlog}
\begin{table}[H]
\centering
\begin{tabular}{|c|c|p{4cm}|p{4cm}|c|c|}
\hline
ID & As a/an & I want to & So that & Estimation & Priority
\\
\hline
1 & user & create an account & I can register for the  service & 3 & 1
\\
\hline
2 & user & login to the system & access to the system and use service & 2 & 1
\\
\hline
5 & user & access to system via a website or an app & a UI makes operation efficient & 5 & 1
\\
\hline
12 & system & set up database and make it nice-designed & store all the user information and file data & 4 & 1
\\
\hline
3 & user & upload files to the server & server can store the files & 5 & 2
\\
\hline
4 & user & download files from server & get files from server & 5 & 2
\\
\hline
8 & user & set up public  space & all the users can access to files & 3 & 2
\\
\hline
9 & user & change  passwords & protect personal information & 3 & 2
\\
\hline
6 & user & delete files to trash and  recover & cut off useless data protect data & 6 & 3
\\
\hline
11 & user & filter and sort files in some order & I can easily find relative files & 6 & 5
\\
\hline
7 & user & search for other users & view  others' profile and files & 5 & 6
\\
\hline
10 & user & add passwords to files & protect data from unauthorized access & 4 & 6
\\
\hline
13 & system & build machanism to  coordinate the simultaneous operation & can solve the conflict generated by different user & 8 & 6
\\
\hline
14 & user & share the files with URL & others can access to files without app & 6 & 7
\\
\hline
15 & user & create new folder and put files in different folders & classify files & 6 & 7
\\
\hline
\end{tabular}
\end{table}
\section{Rough Timetable}
Using the time flow of agile development, we decided to divide the development cycle into five phases. According to the backlog files listed, our schedule is as follows:
\begin{table}[H]
\centering
\begin{tabular}{|c|c|c|p{8cm}|}
\hline
Stage & Period & Task(Sorted by priority) & Remark(*) \\
\hline
1 & -2.8 & 1,2,12* & Complete the design of the database in Task 12 at this stage. \\
\hline
2 & 2.9-2.23 & 12*,3,4,5,8,9 & Continue to complete
unfinished Task 12.
 \\
\hline
3 & 2.24-3.9 & 13*,11,6,7 & Task 13 is the must-done task but at the same time it has certain difficulty so that the completion period can be extended to the fourth stage. \\
\hline
4 & 3.10-3.22 & 13*,10,14,15 & Continue to complete
unfinished Task 13.
\\
\hline
5 & 3.23-3.28 & Debug \& Test &
\\
\hline
\end{tabular}
\end{table}
In particular, Task 1 and 2 are completed by the team members who are primarily responsible for the front end. For Tasks 12 and 13, the team members work in collaborative work mode to complete the back-end code; the rest of the tasks use the division of labor mode to divide the task to different team members.
\section{Initial Progress}
\subsection{FTP server setup and configuration}
Our group purchased Alibaba Cloud’s ESC cloud server, which has Linux operating system. The FTP environment has been built on this server and now we can use DOS commands to connect to this server remotely as well as upload and download files. This laid the foundation for our later work. Because we have the idea of the upload and download functions in the file management system is to use the FTP operation class in Java to achieve these functions.
\subsection{Front-end construction of web application and Android apps}
Based on the backlog and timetable we have developed, the basic front-end construction of the file management system should be complete. These include the most basic feature implementation, such as login and registration on the web application. In the further work, the font end of Android mobile app will be connected with the remote server backend, so that the users can access the server using both the web page and the mobile device. Meanwhile, the design of the database (ER model) has been completed. According to the ER diagram, the database will be built both on the local and server. User and file information is stored in the database. Some of the later functions, for instance, the user setting file password, will use the database for information storage and backup.
\section{Team Roles}
\begin{itemize}
\item Haonan Li is mainly responsible for crucial algorithms coding for solving the conflict problems and meeting organization.
\item Chen Ling is mainly responsible for the development of mobile applications and the development of back-end basic functions.
\item Jiashuo Li mainly takes the charge of design and implementation of the some functionalities of server. He also acts as a user who will propose and update requirements iteratively. In addition, Jiashuo Li collects and summarizes information for each meeting(sprint), updating information on Paper(by Dropbox).
\item Zifei Fu is responsible for the front-end development of web applications, and the implementation of some of the back-end code, as well as the connection to the front and back end. Participate in project goal setting and decision making. Also participate in the code testing for each ‘sprint’.
\item Huikang Liu is responsible for the  partial writing of server-side code. The goal is to implement features such as uploading, downloading, sharing and also cooperating with other team members in the group in dealing with difficult problems encountered.
\item Wenhao Dai is mainly responsible for backend development and interface design.
\end{itemize}
\section{Process and Tools used in collaboration}
\subsection{Process}
Based on face-to-face meetings and real-time communication, our group collaborate efficiently. On the aspect of meetings, we usually hold one or two meetings a week, which last for 1 to 2 hours. We firstly deliver the tasks made in last meeting and then list the main problems exist and manage to find reasonable solutions. At last we determine what needs to be done before the next meeting and assign them individually. Using WeChat and DropPaper, we can communicate with each other conveniently. The record of the meetings, backlogs and other related materials can be uploaded on DropPaper, where we can read and adjust them synchronously.
\subsection{Tools}
\begin{enumerate}
\item WeChat : Daily real-time communication
\item DropPaper : Share important files and records 
\item Git hub : Coding management
\end{enumerate}
\section{Peer Assessment and Conflict}
Individual contributions are discussed by group members, the value is related to the amount of codes written by the team members and the difficulties in implementation. Meetings are held regularly each week which gives each of us a chance to discuss issues and express our opinions. The main reason for the conflict is usually people having different ideas all worth discussing and capable of giving us a deeper understanding of the issues that are being debated. We always handle each conflict with a positive attitude till we arrive on a mutual ground. We will let everyone involved have a say in the group activities and also help group members work around their prejudices. If there is still a conflict that can not be resolved through discussion, We will all vote with the majority carrying the vote. Team members who treat the project negatively will have a certain amount of individual contributions deducted and the specific value will be determined  by the group discussion.
\end{document}